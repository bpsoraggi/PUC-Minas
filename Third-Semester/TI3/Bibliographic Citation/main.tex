\documentclass[12pt]{article}

\usepackage{sbc-template} 
\usepackage{graphicx,url}
\usepackage{url}
\usepackage[brazil]{babel} 
\usepackage[utf8]{inputenc} 
\usepackage[T1]{fontenc}
\usepackage[normalem]{ulem}
\usepackage[hidelinks]{hyperref}

\usepackage[square,authoryear]{natbib}
\usepackage{amssymb} 
\usepackage{mathalfa} 
\usepackage{algorithm} 
\usepackage{algpseudocode} 
\usepackage[table]{xcolor}
\usepackage{array}
\usepackage{titlesec}
\usepackage{mdframed}
\usepackage{listings}
\usepackage{moresize}

\usepackage{amsmath} 
\usepackage{booktabs}

\urlstyle{same}

\newcolumntype{L}[1]{>{\raggedright\let\newline\\\arraybackslash\hspace{0pt}}m{#1}}
\newcolumntype{C}[1]{>{\centering\let\newline\\\arraybackslash\hspace{0pt}}m{#1}}
\newcolumntype{R}[1]{>{\raggedleft\let\newline\\\arraybackslash\hspace{0pt}}m{#1}}

\newcommand\Tstrut{\rule{0pt}{2.6ex}} 
\newcommand\Bstrut{\rule[-0.9ex]{0pt}{0pt}} 
\newcommand{\scell}[2][c]{\begin{tabular}[#1]{@{}c@{}}#2\end{tabular}}

\usepackage[nolist,nohyperlinks]{acronym}

\title{A Presença da Inteligência Artificial na Arte e na Cultura: Uma Revisão Bibliográfica}

\author{
    \small Barbara Hellen Pereira Soraggi\and\\
    \small Bruno Machado Innecco Corrêa\and\\
    \small Camila Hollerbach Pimenta Macedo\and\\
    \small Paulo Henrique Sendas Resende\\
}


\address{PUC Minas - Disciplina de Trabalho Interdisciplinar III}

\begin{document} 
	
	\maketitle
	
	\begin{resumo} 
		Com o objetivo de discutir a presença da IA na arte e na cultura, foi realizado uma revisão bibliográfica a partir de variados artigos com temas relacionados. A inteligência artificial (IA) tem se tornado cada vez mais presente em diversos aspectos de nossas vidas, dentre eles a cultura e a arte. Desde a criação de obras de arte geradas por algoritmos até a utilização da IA em museus e galerias para melhorar a experiência dos visitantes, a tecnologia tem transformado a forma como percebemos e interagimos com a arte. Além disso, a IA também tem sido utilizada para análise e interpretação de dados culturais, como a identificação de padrões em movimentos artísticos e na preferência do público. Embora a presença da IA na cultura e na arte ainda esteja em constante evolução, é inegável que essa tecnologia tem o potencial de transformar significativamente o mundo da arte e da cultura.

	\end{resumo}

 \begin{palavras chaves} 
		IA, inteligência artificial, arte, cultura, museu, algoritmo, futuro, conteúdo, digitalizado, evolução

	\end{palavras chaves}
	
	\section{Introdução}
	\label{sec:introducao}
	
	A presença da IA na cultura e na arte é um fenômeno que tem ganhado cada vez mais destaque nos últimos anos. Uma das formas mais evidentes de sua presença é na criação de obras de arte geradas por algoritmos, que utilizam técnicas de aprendizado de máquina e processamento de linguagem natural para produzir pinturas, músicas, poemas e até mesmo roteiros de filmes. Esse tipo de arte é geralmente conhecido como "arte generativa" e tem levantado questões sobre a criatividade e a autoria, uma vez que as obras são produzidas por algoritmos e não por artistas humanos.

    Além da criação de obras de arte, a IA também tem sido utilizada em museus e galerias para melhorar a experiência dos visitantes. Uma das aplicações mais comuns é a utilização de chatbots para interagir com os visitantes e fornecer informações sobre as obras em exibição. Outra aplicação interessante é a utilização de realidade virtual e aumentada para criar experiências imersivas e interativas.
    
    Outra maneira em que a IA está presente tanto na cultura quanto na arte é na análise e interpretação de dados culturais, que pode ser utilizada para identificar padrões em movimentos artísticos, na preferência do público ou no impacto de determinadas obras na sociedade. Essa análise pode fornecer insights valiosos para curadores de museus, galerias e outras instituições culturais, ajudando-os a tomar decisões mais informadas sobre quais obras exibir e como apresentá-las ao público.

	\section{Metodologia}
	\label{sec:fund_teorica}
	
	A metodologia utilizada para a revisão bibliográfica dos artigos selecionados consistiu na busca e seleção de estudos relevantes sobre o tema proposto. Para isso, foram consultados diferentes bancos de dados acadêmicos, como o Google Scholar, Web of Science e Scopus, utilizando uma combinação de palavras-chave relacionadas à inteligência artificial, direito autoral, propriedade intelectual e arte . Foram considerados artigos publicados em periódicos científicos de renome e também em conferências internacionais relevantes na área. Após a seleção dos artigos, foi realizada uma leitura crítica e sistemática dos textos, buscando identificar pontos relevantes e estabelecer conexões entre as diferentes perspectivas apresentadas. A partir dessa análise, foi possível identificar os principais desafios e oportunidades relacionados ao uso da inteligência artificial em diferentes áreas do conhecimento, bem como as implicações legais e éticas associadas ao tema.
    
    Com o avanço tecnológico, a inteligência artificial tem se mostrado cada vez mais presente em diversas áreas do conhecimento, incluindo a arte. No entanto, a utilização dessa tecnologia também traz desafios e questões complexas, especialmente no que se refere aos direitos autorais e à propriedade intelectual. Por isso, a revisão bibliográfica realizada teve como objetivo examinar os principais debates e argumentos relacionados ao tema, buscando analisar as diferentes perspectivas apresentadas pelos autores dos artigos selecionados. Além disso, a revisão também buscou identificar as principais tendências e direções futuras na área, permitindo uma compreensão mais abrangente e atualizada sobre o tema. Em resumo, a metodologia utilizada buscou fornecer uma visão ampla e integrada sobre a relação entre inteligência artificial e direitos autorais/propriedade intelectual, bem como as implicações 


	\section{Discussão e Resultados}
	\label{sec:trab_relacionados}

    % ----------------------------------- ARTIGO 1 -----------------------------------
    \subsection{Arte e inteligências artificiais: implicações para a criatividade}
	\label{sec:trab_relacionados}
	
	O artigo "Arte e inteligências artificiais: implicações para a criatividade" discute a relação entre a arte e a inteligência artificial (IA), explorando as implicações dessa relação para a criatividade. O autor examina as diferentes formas como a IA pode ser usada na criação artística, desde o auxílio à criatividade humana até a geração autônoma de obras de arte.
    
    O estudo argumenta que a IA pode contribuir para a criatividade humana em várias áreas, como no uso de algoritmos para gerar ideias e inspiração criativa, ou na utilização de técnicas de aprendizado de máquina para analisar grandes quantidades de dados e encontrar padrões que possam ser utilizados na criação artística. No entanto, o autor também alerta para o risco de que a IA possa substituir a criatividade humana, e discute as implicações éticas e filosóficas desse cenário.
    
    O artigo conclui que a IA pode ser uma ferramenta valiosa para a criação artística, mas que é importante manter a criatividade humana no centro do processo criativo. O autor sugere que uma abordagem colaborativa entre humanos e sistemas de IA pode levar a resultados criativos mais ricos e interessantes.

    % ----------------------------------- ARTIGO 2 -----------------------------------
    \subsection{Propriedade Intelectual e Direito Autoral de Produção Autônoma da Inteligência Artificial}
	\label{sec:trab_relacionados}

    O artigo "Propriedade Intelectual e Direito Autoral de Produção Autônoma da Inteligência Artificial" explora questões jurídicas relacionadas à propriedade intelectual e direitos autorais de obras de arte geradas por sistemas de inteligência artificial (IA) autônomos. O estudo aborda as lacunas legais que existem em relação à proteção de direitos autorais para obras criadas por IA, e as implicações jurídicas de autoria, originalidade e direitos autorais nesse contexto.
    
    O artigo destaca a necessidade de uma abordagem equilibrada que considere tanto os interesses dos criadores de obras de arte geradas por IA quanto os interesses da sociedade em geral, e explora possíveis soluções legais para garantir a proteção de direitos autorais nesse contexto. Os autores concluem que a proteção dos direitos autorais de obras de IA autônomas deve ser baseada em critérios que levem em conta a originalidade da obra, a contribuição do criador da IA na geração da obra, e o papel da IA como ferramenta criativa.

    % ----------------------------------- ARTIGO 3 -----------------------------------
    \subsection{A Inteligência Artificial e o Direito do Autor: Uma análise da possibilidade de tutela jurídica para criações intelectuais produzidas com sistemas de inteligência artificial}
	\label{sec:trab_relacionados}

    O artigo "A Inteligência Artificial e o Direito do Autor: Uma análise da possibilidade de tutela jurídica para criações intelectuais produzidas com sistemas de inteligência artificial" discute a questão dos direitos autorais em relação às criações intelectuais produzidas por sistemas de inteligência artificial (IA).
    
    A autora explora a possibilidade de tutela jurídica para obras criadas por IA e discute as questões éticas, legais e sociais envolvidas. Ela argumenta que as obras produzidas por IA podem ser consideradas criativas e originais, mas também levanta questões sobre quem seria o autor ou detentor dos direitos autorais em casos de criações feitas por IA autônoma.
    
    O estudo apresenta diferentes abordagens de proteção dos direitos autorais em relação às obras geradas por IA, como a criação de novas leis de direitos autorais específicas para essas situações, ou a aplicação das leis existentes de direitos autorais a partir da consideração do papel do criador da IA na geração da obra.
    
    No final, a autora conclui que a proteção dos direitos autorais em relação às obras produzidas por IA é um desafio complexo, mas que é importante considerar todas as implicações legais e éticas envolvidas para garantir a justiça e a equidade no cenário atual de rápida evolução tecnológica.


    % ----------------------------------- ARTIGO 4 -----------------------------------
    \subsection{A obra de arte na era da inteligência artificial}
	\label{sec:trab_relacionados}

    A dissertação de mestrado de Carolina Valentim Loch, intitulada "A obra de arte na era da inteligência artificial", apresenta uma reflexão crítica sobre as implicações da inteligência artificial na produção e recepção de arte contemporânea. A autora analisa a relação entre a tecnologia e a arte, bem como as mudanças no papel do artista e do público em um contexto de crescente automatização. Para isso, a pesquisa revisa a literatura existente sobre o tema, apresenta exemplos de obras de arte criadas com o uso da inteligência artificial e discute questões éticas e estéticas relacionadas à arte gerada pelo computador. 
    
    Por fim, a autora do texto conclui que a inteligência artificial pode ser uma ferramenta útil para a produção de arte, mas não deve substituir a criatividade humana e a subjetividade na criação artística. Além disso, a autora destaca a importância de se considerar as implicações éticas e sociais da utilização de tecnologias avançadas na produção de arte.
    
    Esse artigo se relaciona com o nosso tema uma vez que ele apresenta uma reflexão crítica sobre as implicações da inteligência artificial na produção e recepção de arte contemporânea, sendo assim, ele aborda nosso tema e disserta sobre ele de forma mais formal, profunda, explicativa e ditatorial.


    % ----------------------------------- ARTIGO 5 -----------------------------------
    \subsection{As performances criadas por inteligência artificial: o reflexo dos algoritmos na ressurreição digital}
	\label{sec:trab_relacionados}

    O artigo “As performances criadas por inteligência artificial: o reflexo dos algoritmos na ressurreição digital” redigido por Marcos Wachowicz (Universidade Federal do Paraná) e Gustavo Fortunato D'Amico (GEDAI) busca compreender como a utilização  da  tecnologia  de  deepfake influência nos projetos de ressurreição digital. Eles afirmam que esses projetos permitem inserir digitalmente atores já falecidos em novas obras e que, o deepfake consiste em um processo de inteligência artificial em  que  a  máquina  consegue  combinar  materiais  para  criar essas obras  novas.  
    
    Em  seguida,  os autores discutem sobre como  funcionam  os  direitos  dos  intérpretes e como ficam nas situações envolvendo as deepfakes.  Por  fim,  eles analisam como tem sido discutida a titularidade das obras intelectuais criadas por aplicações da inteligência artificial. 
    
    Os autores concluem que no Brasil, a lei não prevê esses tipos de situações, deixando a  titularidade  neste  material  aberta  para  discussões, evidenciando a necessidade de uma adequação  legislativa  para  dirimir  eventuais  conflitos que poderão advir da implementação dessas tecnologias no país
    
    Esse artigo se relaciona com o nosso tema uma vez que ele busca apresentar a relação da inteligência artificial com a arte, sendo ela uma obra, de um modo mais vívido, uma vez que usa a tecnologia para a ressurreição digital. Sendo assim, ele aborda nosso tema e disserta sobre ele de forma mais vívida e explicativa.


    % ----------------------------------- ARTIGO 6 -----------------------------------
    \subsection{Inteligência Artificial e Obras de Arte: Regulação do Direito de Autor Face às Criações Artísticas por Novas Tecnologias}
	\label{sec:trab_relacionados}

    O artigo "Inteligência Artificial e Obras de Arte: Regulação do Direito de Autor Face às Criações Artísticas por Novas Tecnologias" de Isabela do Patrocinio Ceccarelli apresenta alternativas, dentro do sistema jurídico brasileiro, que podem tutelar as novas criações artísticas geradas pela Inteligência Artificial. O texto analisa as novas obras que a inteligência artificial tem sido capaz de gerar, já que se trata de um modelo de tecnologia que vem apresentando um desenvolvimento sem precedentes. Ele também explora como a propriedade intelectual pode ser aplicada nesse contexto, considerando questões como a autoria das obras e a proteção dos direitos do criador original.
    
    O objetivo do trabalho é apresentar as controvérsias que surgem diante de um cenário em que a inteligência artificial é capaz de produzir obras de arte de forma autônoma, analisando a possível tutela dessas obras, sob a ótica do regime jurídico brasileiro de direitos autorais. A autora conclui que há um descompasso entre as categorias jurídicas modernas vis a vis uma sociedade pós-moderna, que propõe novos modelos de criação, conforme uma tecnologia desenfreada.
    
    Esse artigo se relaciona com o nosso tema uma vez que ele analisa as novas obras que a inteligência artificial tem sido capaz de gerar e analisa alternativas possíveis para a tutela dessas obras, sendo assim, ele aborda nosso tema de forma diferente e disserta mais sobre ele.


    % ----------------------------------- ARTIGO 7 -----------------------------------
    \subsection{A Autonomia em Obras Dotadas de Inteligência Artificial}
	\label{sec:trab_relacionados}

    “A autonomia em obras dotadas de inteligência artificial” refere-se à capacidade de criações artísticas e obras de arte geradas por algoritmos ou sistemas de IA de operarem de forma independente e autônoma. Essa questão levanta debates sobre a autoria, criatividade e responsabilidade nessas obras, já que o processo criativo muitas vezes envolve a interação entre o algoritmo e o artista-programador. Além disso, a autonomia dessas obras também traz desafios éticos e legais, como a proteção dos direitos autorais e a definição de limites em relação ao uso de dados e algoritmos.
    
    O tema da autonomia em obras dotadas de inteligência artificial tem uma relação profunda com o futuro da arte. À medida que a tecnologia avança, a IA está se tornando uma ferramenta cada vez mais utilizada por artistas e criadores para explorar novas formas de expressão. A capacidade da IA de gerar criações originais e únicas levanta questões sobre o papel do artista no processo criativo e a redefinição dos conceitos tradicionais de autoria. Além disso, a autonomia das obras de IA também tem o potencial de ampliar os limites da criatividade e da inovação artística, desafiando convenções estabelecidas e oferecendo novas perspectivas e experiências para o público. No entanto, também é necessário enfrentar desafios éticos, legais e sociais para garantir uma abordagem responsável e inclusiva nesse contexto, a fim de equilibrar a colaboração entre humanos e máquinas na criação artística.


    % ----------------------------------- ARTIGO 8 -----------------------------------
    \subsection{Da Construção Textual à Análise de Discurso: do primeiro gesto à explosão de sentidos}
	\label{sec:trab_relacionados}

    Este ensaio busca abordar a seguinte questão: seria possível para um sistema robótico não apenas imitar uma obra de arte, mas também criar uma melodia no estilo de Chopin ou um conto no estilo de Flaubert, combinando características específicas da música de Chopin e do estilo literário de Flaubert? Na primeira parte, exploramos brevemente diversas pesquisas em inteligência artificial que estão sendo realizadas em diferentes partes do mundo, incluindo o Laboratório de Pesquisa de Tecnologia e Criação do Spotify da Sony em Paris, o Departamento de Musicologia da Universidade da Califórnia sob a liderança de David Cope, os esforços da Microsoft na França, a Microsoft Research da Ásia e o Departamento de Pesquisa do Facebook liderado por Yann Le Cun, apresentando argumentos favoráveis e contrários a essa hipótese. Na segunda parte, destacamos as dificuldades que um robô enfrentaria ao tentar imitar o estilo de um escritor, devido à complexidade das palavras reveladas por manuscritos, à sintaxe frequentemente inovadora e à dificuldade de compreensão da comunidade.


    % ----------------------------------- ARTIGO 9 -----------------------------------
    \subsection{Uso de Inteligência Artificial (IA) e Processos Educativos em Museus}
	\label{sec:trab_relacionados}

    O artigo científico de Milene Chiovatto aborda o uso da inteligência artificial (IA) e processos educativos em museus. O estudo explora a aplicação da IA como uma ferramenta inovadora para enriquecer a experiência educacional dos visitantes em museus. Ao utilizar técnicas de IA, como reconhecimento de imagem e processamento de linguagem natural, os museus podem oferecer interações personalizadas, informações contextualizadas e experiências imersivas aos seus visitantes. Além disso, a IA também pode auxiliar na análise de dados para compreender melhor os interesses e preferências dos visitantes, permitindo um planejamento mais eficiente de exposições e atividades educativas. A pesquisa de Chiovatto destaca o potencial da IA na transformação dos processos educativos em museus, proporcionando uma abordagem mais interativa, engajadora e adaptada às necessidades individuais dos visitantes.
    
    O tema do uso de inteligência artificial (IA) e processos educativos em museus se relaciona com o uso da IA na arte, pois ambos exploram o potencial da tecnologia para transformar a experiência dos espectadores e visitantes. Assim como os museus podem utilizar IA para enriquecer a experiência educacional dos visitantes, a IA também pode ser empregada na criação artística para ampliar as possibilidades criativas e a interação com o público. A IA pode ser usada para gerar obras de arte originais, ajudar os artistas na exploração de novas técnicas e estilos, e até mesmo criar instalações interativas e imersivas. Ao incorporar a IA na arte, podemos alcançar novos níveis de expressão e engajamento, permitindo experiências únicas e personalizadas para os apreciadores de arte. Ambos os contextos - uso de IA em museus e uso de IA na arte - estão impulsionando uma evolução significativa na forma como percebemos, interagimos e nos envolvemos com o mundo da arte.


    % ----------------------------------- ARTIGO 10 -----------------------------------
    \subsection{Artificial Intelligence in Cardiology: Concepts, Tools and Challenges}
	\label{sec:trab_relacionados}
    
    O artigo "Artificial Intelligence in Cardiology: Concepts, Tools and Challenges" fornece uma visão geral do uso da inteligência artificial (IA) na cardiologia. O artigo explica que a IA pode ser usada de várias maneiras para melhorar o atendimento ao paciente, incluindo a previsão de resultados do paciente, a identificação de pacientes com alto risco de desenvolver doenças cardíacas e a melhoria da precisão diagnóstica. O artigo também discute as várias ferramentas e técnicas usadas na IA, incluindo aprendizado de máquina, aprendizado profundo e processamento de linguagem natural. O artigo conclui com a citação "O cavalo é quem corre, você precisa ser o jóquei", enfatizando a importância da supervisão e da tomada de decisões humanas no uso da IA na medicina, mais precisamente em cardiologia.
    
    Embora o artigo se concentre na aplicação da inteligência artificial (IA) na cardiologia, ele destaca o potencial da IA para melhorar o atendimento ao paciente e a tomada de decisões em outras áreas da saúde. Essa mesma tecnologia tem sido aplicada em outras áreas, incluindo a arte e a cultura.
    
    A IA tem sido usada para criar novas formas de arte, como a arte generativa, em que algoritmos são usados para gerar imagens, música e outros meios. Além disso, a IA tem sido usada para analisar e catalogar grandes coleções, permitindo que os curadores de arte identifiquem padrões e tendências em obras que antes eram impossíveis de serem identificadas.Em um contexto cultural, a IA tem sido usada para criar novas experiências de aprendizagem e entretenimento. Por exemplo, museus e galerias estão usando IA para fornecer visitas guiadas personalizadas para os visitantes, e a indústria do entretenimento está usando IA para criar personagens e cenários em jogos e filmes que são cada vez mais realistas. Em geral, a presença da IA na arte e na cultura está mudando a forma como as pessoas criam, consomem e interagem com essas formas de expressão, similar ao que ela faz em outras áreas, como na medicina.


    % ----------------------------------- ARTIGO 11 -----------------------------------
    \subsection{Inteligência Artificial (IA) e Arte}
	\label{sec:trab_relacionados}

    O artigo "Inteligência Artificial (IA) e Arte" explora a interseção entre a IA e a arte, destacando como essa tecnologia está sendo usada para criar novas formas de arte e experiências culturais. O artigo começa explicando as diferentes formas em que a IA é usada, incluindo a análise de grandes conjuntos de dados de arte e o uso de IA em experiências culturais.
    
    O artigo também discute algumas das implicações da IA, incluindo a questão da autoria e da originalidade, bem como a preocupação de que a IA possa substituir os artistas humanos. No entanto, o artigo argumenta que a IA é mais uma ferramenta nas mãos dos artistas, permitindo que eles expandam sua criatividade e alcancem novas formas de expressão. O artigo conclui enfatizando a importância de um diálogo contínuo entre a IA e a arte, a fim de explorar as possibilidades criativas que essa tecnologia oferece e encontrar maneiras de incorporá-la de forma ética e responsável no mundo da arte e da cultura.
    
    A presença da inteligência artificial na arte e na cultura está mudando a forma como as pessoas criam, consomem e interagem com essas formas de expressão. É importante continuar explorando as possibilidades criativas que a IA oferece e encontrar maneiras de incorporá-la de forma ética e responsável no mundo da arte e da cultura.


    % ----------------------------------- CONCLUSÃO -----------------------------------
	\section{Conclusão}
	\label{sec:metodologia}
	
	A escassez da informação sobre a presença da IA na arte, apesar da importância deste debate nos dias atuais, nos permite questionar se este fato não se deve ao fato do crescimento acelerado das tecnologias usadas no presente. Mostra também a necessidade de novas pesquisas, que possam subsidiar uma discussão mais ampla e consistente. Mas, pode-se considerar que já houve algum avanço no tópico, como fica evidenciado pela abundância de artigos e postagens feitas a respeito. Como os artigos pesquisados demonstram, a IA pode ser tanto beneficial, quanto também possivelmente maléfica para a sociedade, a arte e a cultura.

    Desta forma, torna-se um fato importante que devemos persistir em discutir esse tópico e aprofundar nosso conhecimento a respeito de leis e métodos de implementações das IAs. É preciso valorizar os aspectos positivos e otimistas decorrentes do uso destas novas tecnologias, mas também é preciso efeturar o uso com prudência.
	
    \bibliographystyle{apalike}
    \bibliography{references}

    VENANCIO JÚNIOR, S. J. Arte e inteligências artificiais: implicações para a criatividade. ARS (São Paulo), v. 17, n. 35, p. 183–201, 12 maio 2019.

    DIVINO, S. B. S.; MAGALHÃES, R. A. Propriedade intelectual e direito autoral de produção autônoma da inteligência artificial. Revista de Direitos e Garantias Fundamentais, v. 21, n. 1, p. 167–192, 2020.

    AMARAL, A. C. S. G. DO. A INTELIGÊNCIA ARTIFICIAL E O DIREITO DO AUTOR: Uma análise da possiblidade de tutela jurídica para criações intelectuais produzidas com sistemas de inteligência artificial. Res Severa Verum Gaudium, v. 5, n. 1, 25 out. 2020.

    UNIVERSIDADE TECNOLÓGICA FEDERAL DO PARANÁ DEPARTAMENTO ACADÊMICO DE LINGUAGEM E COMUNICAÇÃO PROGRAMA DE PÓS-GRADUAÇÃO EM ESTUDOS DE LINGUAGENS CAROLINA VALENTIM LOCH A OBRA DE ARTE NA ERA DA INTELIGÊNCIA ARTIFICIAL DISSERTAÇÃO CURITIBA 2021. [s.l: s.n.].

    WACHOWICZ, M.; D’AMICO, G. F. As performances criadas por inteligência artificial: o reflexo dos algoritmos na ressurreição digital. Revista Rede de Direito Digital, Intelectual e Sociedade, v. 2, n. 3, p. 17–37, 12 set. 2022.

    CECCARELLI, I. DO P. Inteligência artificial e obras de arte: regulação do direito de autor face às criações artísticas por novas tecnologias. bibliotecadigital.fgv.br, 26 abr. 2021.  SÃO PAULO 2021. [s.l: s.n.].

    KERINSKA, N. T. A autonomia em obras dotadas de inteligência artificial. PÓS: Revista do Programa de Pós-graduação em Artes da EBA/UFMG, v. 10, n. 19, p. 42–58, 27 maio 2020.

    v. 23 n. 2 (2020): Da Construção Textual à Análise de Discurso: do primeiro gesto à explosão de sentidos | Signum: Estudos da Linguagem.

    CHIOVATTO, M. WATSON, USO DE INTELIGÊNCIA ARTIFICIAL (AI) E PROCESSOS EDUCATIVOS EM MUSEUS. Revista Docência e Cibercultura, v. 3, n. 2, p. 217–230, 1 set. 2019.

    SOUZA FILHO, E. M. DE et al. Artificial Intelligence in Cardiology: Concepts, Tools and Challenges - “The Horse is the One Who Runs, You Must Be the Jockey”. Arquivos Brasileiros de Cardiologia, n. AHEAD, 2019.

    WILLEMART, P. L. Inteligência Artificial (IA) e Arte. Signum: Estudos da Linguagem, v. 23, n. 2, p. 10, 18 mar. 2021.

    
	
\end{document}
